\chapter{场和共形不变性}

量子场论,要从场的局域函数,Lagrangian密度或作用量开始,作量子化(正则量子化或路径积分量子化)。场的关联函数由含源的配分函数计算,对称性在其中扮演了重要角色。 能动张量守恒来自Poincaré不变性,电荷守恒来自复场相位变换下的不变性。虽然这些对应全局的变换,但它们的局域版本,规范变换,通过Ward-Takahashi恒等式,很大程度上限制了关联函数的结构。本章解释共形不变性在场论中的意义。

\section{作用量和守恒量}
考虑经典场论中的共形不变性。经典理论中的共形对称性,是说作用量在共形变换下不变。2.1和2.2节是对场论的回顾。
\section{场的共形变换}
考虑 $d $维时空中 N 分量场 $\phi(x)=\left(\phi^{a}(x)\right)$ ( $a=1,\cdots,N $)在共形变换下的行为。首先,我们要考虑的场论是相对论理论,具有Poincaré不变性。先看Lorentz变换
\begin{equation}
	x^{\mu} \rightarrow x^{\prime \mu}=\Lambda^{\mu} {}_{\nu}x^{\nu}
\end{equation}
下的行为。这时场变换为:
\begin{equation}
	\phi^{a}(x) \rightarrow \phi^{\prime a}\left(x^{\prime}\right)=D(\Lambda)^{a}{ }_{b} \phi^{b}(x)
\end{equation}

从Lorentz变换 $\Lambda $到 $N\times N $矩阵 $D(\Lambda) $的映射: $\Lambda \mapsto D(\Lambda)$ ,是Lorentz群的 N 维表示。也就是说,

对恒等元 $1$ ,有 $D(1)=1_N $( $1_N $是$ N$ 阶恒等矩阵);
对Lorentz变换的复合 $\Lambda_3=\Lambda_2\Lambda_1$ ,有
\begin{equation}
	D\left(\Lambda_{3}\right)=D\left(\Lambda_{2}\right) D\left(\Lambda_{1}\right)
\end{equation}
对无穷小Lorentz变换 $\Lambda^{\mu}{}_{\nu}=\delta_{\nu}^{\mu}+\omega^{\mu}{}_{\nu} $,表示矩阵$ D(\Lambda) $是
\begin{equation}
	D(\Lambda)^{a}{}_{b}=\delta_{b}^{a}-\frac{i}{2} \omega_{\mu \nu}\left(S^{\mu \nu}\right)^{a}{}_{b}
\end{equation}
$N$ 阶矩阵$ S_{\mu \nu}$ 满足Lorentz代数的对易关系。

在相对论场论中,作用量$ S$ 在Lorentz变换下不变,它是基本的场 $\phi( x)$ 及其导数 $\partial_\mu \phi(x)$ 的函数,Lagrangian密度 $\mathcal{L}\left(\phi, \partial_{\mu} \phi\right) $的积分:
\begin{equation}
	S=\int d^{d} x \mathcal{L}\left(\phi(x), \partial_{\mu} \phi(x)\right)
\end{equation}
出现在Lagrangian中的场,由Lorentz群的不可约表示分类。例如,标量场 $\phi(x) $在Lorentz变换下不变:
\begin{equation}
\phi^{\prime}\left(x^{\prime}\right)=\phi(x)
\end{equation}
向量场$ A_\mu(x)$变换为
\begin{equation}
	A_{\mu}^{\prime}\left(x^{\prime}\right)=\Lambda_{\mu}{}^{\nu} A_{\nu}(x)
\end{equation}
属于Lorentz群的$ N$ 维表示的场 $\phi_a(x)$ ,按 (2.2) 变换,点 $x$ 处场变化了
\[\phi'^{a}(x)-\phi^{a}(x)=D(\Lambda)^{a}{ }_{b} \phi^{b}\left(\Lambda^{-1} x\right)-\phi^{a}(x)\]
对无穷小变换,场的无穷小变化(称为Lie导数) $\delta_0\phi^a=\phi'^{a}(x)-\phi^a(x) $是
\begin{equation}
\begin{aligned} \delta_{0} \phi^{a}(x)&=\omega_{\mu \nu} \frac{1}{2}\left(x^{\mu} \partial^{\nu}-x^{\nu} \partial^{\mu}\right) \phi^{a}(x)-\frac{i}{2} \omega_{\mu \nu}\left(S^{\mu \nu}\right)^{a}{}_{b} \phi^{b}(x)\\ &=-\frac{i}{2} \omega_{\mu \nu}\left(L^{\mu \nu}\right)^{a} {}_{b}\phi^{b}(x) \end{aligned}
\end{equation}
这里,$ L^{\mu\nu} $是同时包含了坐标和场变换的Lorentz变换生成元:
\begin{equation}
	L^{\mu \nu}=i\left(x^{\mu} \partial^{\nu}-x^{\nu} \partial^{\mu}\right) 1_{N}+S^{\mu \nu}
\end{equation}
其满足Lorentz代数。

对无穷小共形变换,生成元对场 $\phi(x)$ 的作用也可类似地得到,Lie导数是
\begin{equation}
	\delta_{0} \phi(x)=\left(-i \epsilon^{\mu} P_{\mu}-\frac{i}{2} \omega^{\mu \nu} L_{\mu \nu}-i \epsilon D-i b^{\mu} K_{\mu}\right) \phi(x)
\end{equation}

平移$P_\mu $,Lorentz变换 $L_{\mu\nu} $,标度变换 $D$ 和特殊共形变换 $K_\mu $的生成元对场 $\phi(x)$ 的作用是
\begin{align} &P_{\mu} \phi(x)=-i \partial_{\mu} \phi(x)\\ &L_{\mu \nu} \phi(x)=i\left(x_{\mu} \partial_{\nu}-x_{\nu} \partial_{\mu}\right) \phi(x)+S_{\mu \nu} \phi(x)\\ &D \phi(x)=-i x^{\mu} \partial_{\mu} \phi(x)+i \Delta \phi(x)\\ &K_{\mu} \phi(x)=i\left(x^{2} \partial_{\mu}-2 x_{\mu} x^{\nu} \partial_{\nu}\right) \phi(x) +\left(\kappa_{\mu}+2 x_{\mu} i \Delta-x^{\nu} S_{\mu \nu}\right) \phi(x) \end{align}
这里,$ \Delta$ , $\kappa_\mu$ 是$ N $阶矩阵,满足对易关系
\begin{align} &\left[\Delta, S_{\mu \nu}\right]=0 \\ &\left[\Delta, \kappa_{\mu}\right]=\kappa_{\mu} \\ &\left[\kappa_{\mu}, \kappa_{\nu}\right]=0\\ &\left[\kappa_{\rho}, S_{\mu \nu}\right]=i\left( \eta_{\rho \mu} \kappa_{\nu}-\eta_{\rho \nu} \kappa_{\mu}\right) \\ &\left[S_{\mu \nu}, S_{\rho \sigma}\right]=i\left(\eta_{\nu \rho} S_{\mu \sigma}+\eta_{\mu \sigma} S_{\nu \rho}-\eta_{\mu \rho} S_{\nu \sigma}-\eta_{\nu \sigma} S_{\mu \rho}\right) \end{align}

如果$ \phi(x)$ 属于Lorentz群的不可约表示,那么所有与 $S_{\mu\nu} $对易的生成元正比于恒等矩阵。于是$ \Delta$ 正比于恒等矩阵,系数仍记作 $\Delta $。从而,由对易关系 (2.16) ,有 $\kappa_\mu=0 $。因此,在共形不变的场论中,基本的场除由Lorentz群的表示分类以外,还由标度变换表示中的参数 $\Delta$ 分类。$ \Delta$ 称为场 $\phi $的\textbf{共形维数}。

如果$ \phi(x)$ 是标量场, $S_{\mu\nu}=0$ ,有限的共形变换 $x\to x^\prime$ 下,标量场变为
\begin{equation}
		\phi^{\prime}\left(x^{\prime}\right)=\left|\frac{\partial x^{\prime}}{\partial x}\right|^{-\frac{\Delta}{d}} \phi(x)
\end{equation}
$| \partial x^{\prime}/\partial x |$是共形变换的Jacobian。对Lorentz变换则是 $| \partial x^{\prime}/\partial x |=1 $,因此得到 (2.6) 。由共形变换下度规的变换规则 (1.14) ,有
\begin{equation}
	\Omega^{2}(x) \eta_{\mu \nu} \frac{\partial x^{\prime \mu}}{\partial x^{\rho}} \frac{\partial x^{\prime \nu}}{\partial x^{\sigma}}=\eta_{\rho \sigma}
\end{equation}
于是取行列式,得到Jacobian是
\begin{equation}
	\left|\frac{\partial x^{\prime}}{\partial x}\right|=\Omega^{-d}(x)
\end{equation}

特别地,对标度变换 $x'^\mu=\alpha x^\mu$ , $\Omega=\alpha^{-1}$ ,因此场 $\phi(x)$ 变换为
\begin{equation}
\phi^{\prime}\left(x^{\prime}\right)=\alpha^{-\Delta} \phi(x)
\end{equation}
对特殊共形变换 (1.59) ,
\begin{equation}
	\Omega^{2}(x)=\left(x^{\prime 2}\right)^{2}=\left(1-2 b x+b^{2} x^{2}\right)^{-2}
\end{equation}
场变换为
\begin{equation}
		\phi^{\prime}\left(x^{\prime}\right)=\frac{1}{\left(1-2 b x+b^{2} x^{2}\right)^{\Delta}} \phi(x)
\end{equation}

标度变换 $x'^\mu=\alpha x^\mu$ 表示各处长度均匀地变为原来的 $\alpha $倍。因为质量量纲是长度的倒数量纲,具有质量的$ \Delta$ 次幂量纲的量在这标度变换下变为原来的 $\alpha^{-\Delta}$ 倍。于是由 (2.23) , $\phi(x)$ 具有质量的 $\Delta$ 次幂量纲。

\subsection{共形变换和守恒量}

简单起见,本节考虑标量场。考虑标量场$ \phi(x)$ 在区域$ V $中的作用量 $S=\int_{V} d^{d} x \mathcal{L}\left(\phi, \partial_{\mu} \phi\right) $,在无穷小坐标变换
\begin{equation}
	x^{\mu} \rightarrow x^{\prime \mu}=x^{\mu}+\delta x^{\mu}(x)
\end{equation}
下的变化。这一坐标变换下,假定场变换为:$ \phi(x) \rightarrow \phi^{\prime}(x)=\phi(x)+\delta_0 \phi(x)$ 。Lagrangian的变化,要从 $\mathcal{L} $是标量,以及场的变化会引起它变化,这两方面来考虑,结果是
\begin{equation}
		\delta \mathcal{L}=\frac{\partial \mathcal{L}}{\partial \phi} \delta_{0} \phi+\frac{\partial \mathcal{L}}{\partial\left(\partial_{\mu} \phi\right)} \delta_{0}\left(\partial_{\mu} \phi\right)+\delta x^{\mu} \partial_{\mu} \mathcal{L}
\end{equation}
利用 $\delta_0(\partial_\mu \phi)=\partial_\mu(\delta_0 \phi) $,这可改写成
\begin{equation}
	\delta \mathcal{L}=\left\{\frac{\partial \mathcal{L}}{\partial \phi}-\partial_{\mu}\left(\frac{\partial \mathcal{L}}{\partial\left(\partial_{\mu} \phi\right)}\right)\right\} \delta_{0} \phi+\partial_{\mu}\left(\frac{\partial \mathcal{L}}{\partial\left(\partial_{\mu} \phi\right)} \delta_{0} \phi\right)+\delta x^{\mu} \partial_{\mu} \mathcal{L}
\end{equation}
此外, $d$ 维体积元变为:
\begin{equation}
	d^{d} x \rightarrow d^{d} x^{\prime}=d^{d} x\left|\frac{\partial x^{\prime}}{\partial x}\right|
\end{equation}
这里,Jacobian是
\begin{equation}
	\left|\frac{\partial x^{\prime}}{\partial x}\right|=\left|\delta_{\mu \nu}+\partial_{\mu} \delta x_{\nu}\right|=1+\partial_{\mu} \delta x^{\mu}
\end{equation}
这两方面放在一起,得到作用量变化了
\begin{equation}
	\begin{aligned} \delta S=&\int_{V} d^{d} x\Bigg\{(\partial_{\mu} \delta x^{\mu}) \mathcal{L}+\delta x^{\mu} \partial_{\mu} \mathcal{L} \\&+\left[\frac{\partial \mathcal{L}}{\partial \phi}-\partial_{\mu}\left(\frac{\partial \mathcal{L}}{\partial\left(\partial_{\mu} \phi\right)}\right)\right] \delta_{0} \phi+\partial_{\mu}\left(\frac{\partial \mathcal{L}}{\partial\left(\partial_{\mu} \phi\right)} \delta_{0} \phi\right)\Bigg\}\end{aligned}
\end{equation}
利用Euler-Lagrange方程
\begin{equation}
	\frac{\partial \mathcal{L}}{\partial \phi}-\partial_{\mu}\left(\frac{\partial \mathcal{L}}{\partial\left(\partial_{\mu} \phi\right)}\right)=0
\end{equation}
作用量的变化 $\delta S$ 可改写成全微分形式:
\begin{equation}
	\delta S=\int_{V} d^{d} x \partial_{\mu}\left\{\mathcal{L} \delta x^{\mu}+\frac{\partial \mathcal{L}}{\partial\left(\partial_{\mu} \phi\right)} \delta_{0} \phi\right\}
\end{equation}
现在将无穷小变换的参数换成 $\epsilon $:
$$
	\delta x^{\mu}=\frac{\delta x^{\mu}}{\delta \epsilon} \delta \epsilon, \quad \delta_{0} \phi=\frac{\delta_{0} \phi}{\delta \epsilon} \delta \epsilon
$$
于是 (2.33) 成为
\begin{equation}
		\delta S=\int_{V} d^{d} x \partial_{\mu}\left\{\mathcal{L} \frac{\delta x^{\mu}}{\delta \epsilon}+\frac{\partial \mathcal{L}}{\partial\left(\partial_{\mu} \phi\right)} \frac{\delta_{0} \phi}{\delta \epsilon}\right\} \delta \epsilon
\end{equation}
因为 $V$ 可以任意小,如果作用量在这变换下不变, $\delta S=0 $,那么流
\begin{equation}
	j^{\mu}=-\mathcal{L} \frac{\delta x^{\mu}}{\delta \epsilon}-\frac{\partial \mathcal{L}}{\partial\left(\partial_{\mu} \phi\right)} \frac{\delta_{0} \phi}{\delta \epsilon}
\end{equation}
满足 $\partial^\mu j_\mu=0$ , $j^0 $的$ d-1$ 维空间积分
\begin{equation}
	Q=\int \mathrm{d}^{d-1} x j^{0}
\end{equation}
无关时间,也就是说,是守恒量。因此,如果作用量 (2.5) 在连续变换下不变,就存在守恒量,这称为\textbf{Noether定理}。

例如,考虑全局平移 $\delta x^{\mu}=\epsilon^{\mu}$ ,这时
\begin{equation}
	\delta_{0} \phi(x)=\phi^{\prime}(x)-\phi(x)=\phi(x-\epsilon)-\phi(x)=-\epsilon^{\mu} \partial_{\mu} \phi(x)
\end{equation}
那么作用量变化了
\begin{equation}
	\delta S=\int_{V} d^{d} x \partial_{\mu}\left\{\mathcal{L} \delta_{\nu}^{\mu}-\frac{\partial \mathcal{L}}{\partial\left(\partial_{\mu} \phi\right)} \partial_{\nu} \phi\right\} \epsilon^{\nu}
\end{equation}
如果平移后作用量不变,张量
\begin{equation}
	T_{\mu \nu}=\frac{\partial \mathcal{L}}{\partial\left(\partial^{\mu} \phi\right)} \partial_{\nu} \phi-\eta_{\mu \nu} \mathcal{L}
\end{equation}
满足
\begin{equation}
	\partial^{\mu} T_{\mu \nu}=0
\end{equation}
$T^{0\nu} $的空间积分
\begin{equation}
	P^{\nu}=\int d^{d-1} x T^{0 \nu}
\end{equation}
守恒。 $P^\mu$ 是代表场的能量和动量的$ d$ 维向量。 $T_{\mu \nu} $称为能动张量。

在相对论场论中,还有对应Lorentz变换的守恒流。对无穷小Lorentz变换 $\delta x^{\mu}=\omega^{\mu \nu} x_{\nu}$ ,令 $\epsilon=\omega^{\rho \sigma} $,上标 $\rho,\sigma $固定。无穷小变换下场的变化 (2.8) 代入 (2.35) ,得到对应Lorentz变换的流
\begin{equation}
	\begin{aligned} J^{\mu \rho \sigma}&=-\frac{\partial \mathcal{L}}{\partial\left(\partial_{\mu} \phi\right)}\left(-\frac{i}{2} M^{\rho \sigma} \phi\right)-\mathcal{L} \frac{1}{2}\left(\eta^{\mu \rho} x^{\sigma}-\eta^{\mu \sigma} x^{\rho}\right)\\ &=-\frac{1}{2}\left(x^{\rho} T^{\mu \sigma}-x^{\sigma} T^{\mu \rho}\right) \end{aligned}
\end{equation}

在共形不变的场论中,还有对应标度变换和特殊共形变换的流。对无穷小标度变换 $\delta x^{\mu}=(1+\epsilon) x^{\mu}$ ,流是
\begin{equation}
	\begin{aligned} j_{D}^{\mu}&=-x^{\mu} \mathcal{L}-\frac{\partial \mathcal{L}}{\partial\left(\partial_{\mu} \phi\right)}\left(-x^{\nu} \partial_{\nu} \phi-\Delta \phi\right)\\&=x_{\nu} T^{\mu \nu}+\Delta \frac{\partial \mathcal{L}}{\partial\left(\partial_{\mu} \phi\right)} \phi \end{aligned}
\end{equation}
这称为标度变换流。对应特殊共形变换的守恒流也可类似地得到。

例如,考虑 $d $维中质量为$ m $的自由标量场 $\phi(x) $。Lagrangian是
\begin{equation}
\mathcal{L}=\frac{1}{2}\left(\partial_{\mu} \phi \partial^{\mu} \phi-m^{2} \phi^{2}\right)
\end{equation}
$\phi(x) $具有质量的 $\Delta=(d-2)/2$ 次幂量纲。由 (2.39) ,能动张量是
\begin{equation}
	T_{\mu \nu}=\partial_{\mu} \phi \partial_{\nu} \phi-\eta_{\mu \nu} \mathcal{L}
\end{equation}
从Klein-Gordon方程 $\left(\partial^{\mu} \partial_{\mu}+m^{2}\right) \phi=0 $,可直接验证 $\partial^{\mu} T_{\mu \nu}=0$ 。标度变换流是
\begin{equation}
	j_{D}^{\mu}=\partial^{\mu} \phi\left(x^{\nu} \partial_{\nu}\phi+\Delta \phi\right)-x^{\mu} \mathcal{L}
\end{equation}
它的散度是
\[\partial_{\mu} j_{D}^{\mu}=m^{2} \phi^{2}\]

因此, $m $非零时理论在标度变换下并非不变。无质量标量场的理论是共形不变的。一般地,含质量等有量纲参数的作用量,不是共形不变的。不过,经典的共形不变性不一定意味着量子的共形不变性。例如,四维Yang-Mills理论在经典情形,耦合常数不具有质量量纲,但在量子情形,标度不变性被破坏了。称为 $\mathcal{N}=4$ 超对称Yang-Mills理论的规范理论则是共形不变的。

\section{路径积分和关联函数}

计算场的关联函数是场论中的基本问题。本书中无法详细解释,请参考标准的场论教材。这里只总结一下对之后讨论必要的内容。

场 $\varphi(x) $组成的算符 $A_{1}\left(x_{1}\right), \cdots, A_{N}\left(x_{N}\right) $的关联函数,用路径积分定义为
\begin{equation}
	\left\langle 0\left|T\left(A_{1}\left(x_{1}\right) \cdots A_{N}\left(x_{N}\right)\right)\right| 0\right\rangle=\frac{1}{Z} \int D \varphi A_{1}\left(x_{1}\right) \cdots A_{N}\left(x_{N}\right) e^{i S[\varphi]}
\end{equation}
这里, $|0\rangle $是对应真空的右矢,$ T(...)$ 指编时, $S[\varphi]$ 是作用量。$ Z$ 是配分函数
\begin{equation}
	Z=\int D \varphi e^{i S[\varphi]}
\end{equation}
考虑场 $\varphi(x) $的$ N $点关联函数
\begin{equation}
		\left\langle 0\left|T\left(\varphi\left(x_{1}\right) \cdots \varphi\left(x_{N}\right)\right)\right| 0\right\rangle=\frac{1}{Z} \int D \varphi \varphi\left(x_{1}\right) \cdots \varphi\left(x_{N}\right) e^{i S[\varphi]}
\end{equation}
这可写成含源配分函数
\begin{equation}
		Z[J]=\int D \varphi e^{i S[\varphi]+i \int d^{d} x J(x) \varphi(x)}
\end{equation}
关于源 $J(x) $求泛函导数:\footnote{[原书中应该漏掉了$i$]}
\begin{equation}
		\left\langle 0\left|T\left(\varphi\left(x_{1}\right) \cdots \varphi\left(x_{N}\right)\right)\right| 0\right\rangle=\left.\frac{1}{Z} \frac{\delta}{i\delta J\left(x_{1}\right)} \cdots \frac{\delta}{i\delta J\left(x_{N}\right)} Z[J]\right|_{J=0}
\end{equation}

我们来计算质量为 $m$ 的自由标量场的关联函数。作用量是
\begin{equation}
		S[\varphi]=\int d^{d} x\left\{\frac{1}{2} \partial_{\mu} \varphi \partial^{\mu} \varphi-\frac{1}{2} m^{2} \varphi^{2}\right\}
\end{equation}
引入Feynman传播子
\begin{equation}
		D_{F}(x-y)=\int \frac{d^{d} k}{(2 \pi)^{d}} \frac{e^{-i k(x-y)}}{k^{2}-m^{2}+i \epsilon}
\end{equation}
这是Klein-Gordon算符$ \partial^2+m^2 $的Green函数,满足 $\left(\partial^{2}+m^{2}\right) D_{F}(x-y)=-i \delta^{d}(x-y) $,$ \delta^{d}(x-y)$ 是 $d$ 维delta函数。
作换元
\begin{equation}
	\varphi(x)=\varphi^{\prime}(x)-i \int d^{d} y D_{F}(x-y) J(y)
\end{equation}
可得
\begin{equation}
	Z[J]=N^{\prime} \exp \left(-\frac{1}{2} \int d^{d} x d^{d} y J(x) D_{F}(x-y) J(y)\right)
\end{equation}
$N' $是常数。代入 (2.51) , $N $为奇数时 $\left\langle 0\left|T\left(\varphi\left(x_{1}\right) \cdots \varphi\left(x_{N}\right)\right)\right| 0\right\rangle=0 $, $N $为偶数时
\begin{equation}
		\left\langle 0\left|T\left(\varphi\left(x_{1}\right) \cdots \varphi\left(x_{N}\right)\right)\right| 0\right\rangle=\sum D_{F}\left(x_{i_{1}}-x_{i_{2}}\right) \cdots D_{F}\left(x_{i_{N-1}}-x_{i_{N}}\right)
\end{equation}
这里是对从 $\{1,\cdots,N\} $中取指标对 $\left(i_{1}, i_{2}\right), \cdots,\left(i_{N-1}, i_{N}\right) $的所有可能方式求和。特别地, $N=2$ 时,
\begin{equation}
		\left\langle 0\left|T\left(\varphi\left(x_{1}\right) \varphi\left(x_{2}\right)\right)\right| 0\right\rangle=D_{F}\left(x_{1}-x_{2}\right)
\end{equation}
自由标量场的 $N $点关联函数可写成两点函数的乘积,这称为Wick定理。

例如, $d$ 维Euclid空间中无质量标量场的传播子满足
\begin{equation}
		\partial_{\mu} \partial^{\mu} D(x)=-\delta^{d}(x)
\end{equation}
$\partial^\mu \partial_\mu $是 $d $维Laplacian。这一方程的解可用极坐标表示。径向坐标记作 $r$ ,$ d\geq 3 $时有
\begin{equation}
		D(x)=\frac{1}{(d-2) \Omega_{d}} \frac{1}{r^{d-2}}
\end{equation}
这里,$ \Omega_{d}=2 \pi^{d / 2} / \Gamma(d / 2)$ 是 $d$ 维单位球的表面积,$ \Gamma(d) $是gamma函数。 $d=2 $时有
\begin{equation}
	D(x)=\frac{1}{2 \pi} \ln \left(\frac{1}{r}\right)
\end{equation}
这个传播子在第3章将多次遇到。
\section{共形Ward恒等式}
不同于自由场的情形,有相互作用时关联函数很难精确求出,需要用一些近似方法,例如微扰论。不过,如果理论具有高度的对称性,关联函数的形式是可以确定的。在共形不变的场论中,关联函数的形式在多大程度上是可以确定的呢?Polyakov\footnote{A. M. Polyakov, JETP Lett. 12 (1970) 381 [Pisma Zh. Eksp. Teor. Fiz. 12 (1970) 538].}研究过这个问题。简单起见,接下来略去关联函数中的编时记号。

在有限共形变换 $x\to x' $下,按 (2.20) 变换的场称为\textbf{准初级场}\footnote{局域共形变换下这样变换则称为\textbf{初级场(primary field)},见第3章。}(quasi primary field)。令 $\phi_i(x) (i=1,\cdots,N )$是共形维数为$ \Delta_i$ 的准初级场,也就是说,在共形变换下变为
\begin{equation}
	\phi_{i}^{\prime}\left(x^{\prime}\right)=\left|\frac{\partial x^{\prime}}{\partial x}\right|^{-\frac{\Delta_{i}}{d}} \phi_{i}(x)
\end{equation}
考虑这个准初级场的关联函数
\begin{equation}
	\left\langle\phi_{1}\left(x_{1}\right) \cdots \phi_{N}\left(x_{N}\right)\right\rangle=\frac{1}{Z} \int D \varphi \phi_{1}\left(x_{1}\right) \cdots \phi_{N}\left(x_{N}\right) e^{i S[\varphi]}
\end{equation}
$\varphi $是用于定义 $\phi_i(x)$ 的基础的场。

假定共形变换下作用量 $S[\varphi]$ 不变,路径积分的测度 $D\varphi $也不变。共形变换下 $\varphi$ 变为 $\varphi' $。在关联函数 (2.62) 中,将路径积分的变量从$ \varphi$ 换成$ \varphi'$ ,我们有
\begin{equation}
	\int D \varphi^{\prime} \phi_{1}^{\prime}\left(x_{1}\right) \cdots \phi_{N}^{\prime}\left(x_{N}\right) e^{i S\left[\varphi^{\prime}\right]}=\int D \varphi \phi_{1}\left(x_{1}\right) \cdots \phi_{N}\left(x_{N}\right) e^{i S[\varphi]}
\end{equation}
再将点 $x_i$ 换成 $x_i' $,由作用量和测度的共形不变性有
\begin{equation}
	\left\langle\phi_{1}^{\prime}\left(x_{1}^{\prime}\right) \cdots \phi_{N}^{\prime}\left(x_{N}^{\prime}\right)\right\rangle=\left\langle\phi_{1}\left(x_{1}^{\prime}\right) \cdots \phi_{N}\left(x_{N}^{\prime}\right)\right\rangle
\end{equation}
代入 (2.61) ,得到
\begin{equation}
	\left\langle\phi_1\left(x_{1}\right) \cdots \phi_{N}\left(x_{N}\right)\right\rangle=\prod_{i=1}^{N}\left|\frac{\partial x^{\prime}}{\partial x}\right|_{x=x_{i}}^{\frac{\Delta_{i}}{d}}\left\langle\phi\left(x_{1}^{\prime}\right) \cdots \phi_{N}\left(x_{N}^{\prime}\right)\right\rangle
\end{equation}
这里, $|\partial x'/\partial x|_{x=x_i} $是 $x_i $处的Jacobian。准初级场关联函数满足的关系 (2.65) ,称为\textbf{共形Ward恒等式}。这个关系能在多大程度上确定关联函数的形式呢?

首先考虑Poincaré变换 $x \rightarrow x^{\prime}=\Lambda x+a $,这时 $|\partial x'/\partial x|=1$ ,关联函数不变:
\begin{equation}
	\left\langle\phi_1\left(x_{1}\right) \cdots \phi_{N}\left(x_{N}\right)\right\rangle=\left\langle\phi\left(x_{1}^{\prime}\right) \cdots \phi_{N}\left(x_{N}^{\prime}\right)\right\rangle
\end{equation}
关联函数在均匀的平移(相当于 $\Lambda=1 $时)下不变,说明对关联函数来说,没有特殊的坐标原点,它是差 $x_i-x_j$ 的函数。此外,在Lorentz变换 $x^{\prime}=\Lambda x $下不变,说明它是差 $x_i-x_j $的长度,也就是
\begin{equation}
	r_{i j} \equiv\left|x_{i}-x_{j}\right|=\sqrt{\left(x_{i}-x_{j}\right)^{2}}
\end{equation}
的函数。我们从Poincaré不变性,发现$ N $点关联函数可写成
\begin{equation}
	\left\langle\phi\left(x_{1}\right) \cdots \phi_{N}\left(x_{N}\right)\right\rangle=G^{(N)}\left(\left\{r_{i j}\right\}\right)
\end{equation}
这里,$ \{r_{i j}\} $表示独立变量 $r_{ij} (i<j )$的集合。接下来从共形不变性,还能进一步限制关联函数。我们考察具体的例子。
\subsection{两点函数}
考虑两点函数
\begin{equation}
	\left\langle\phi_{1}\left(x_{1}\right) \phi_{2}\left(x_{2}\right)\right\rangle=G^{(2)}(r)
\end{equation}
这里 $r=r_{12}$ 。共形变换 $x\to x' $的Jacobian是 $|\partial x'/\partial x|=\Omega^{-d}(x) $,共形Ward恒等式成为
\begin{equation}
	\left\langle\phi_{1}\left(x_{1}\right) \phi_{2}\left(x_{2}\right)\right\rangle=\Omega^{-\Delta_{1}}\left(x_{1}\right) \Omega^{-\Delta_{2}}\left(x_{2}\right)\left\langle\phi_{1}\left(x_{1}^{\prime}\right) \phi_{2}\left(x_{2}^{\prime}\right)\right\rangle
\end{equation}

标度变换 $x^{\mu} \rightarrow x^{\prime \mu}=\alpha x^{\mu}$ 下,Jacobian是 $|\partial x'/\partial x|=\alpha^d $,两点函数 (2.69) 的共形Ward恒等式成为
\begin{equation}
	\left\langle\phi_{1}\left(x_{1}\right) \phi_{2}\left(x_{2}\right)\right\rangle=\alpha^{\Delta_{1}+\Delta_{2}}\left\langle\phi_{1}\left(x_{1}^{\prime}\right) \phi_{2}\left(x_{2}^{\prime}\right)\right\rangle
\end{equation}
用$ G^{(2)}(r)$ 表示是
\begin{equation}
	G^{(2)}(r)=\alpha^{\Delta_{1}+\Delta_{2}} G^{(2)}(\alpha r)
\end{equation}\
令$\alpha r=r_0 $,右边用 $r$ 和 $r_0 $改写成
\begin{equation}
	G^{(2)}(r)=\frac{r_{0}^{\Delta_{1}+\Delta_{2}} G^{(2)}\left(r_{0}\right)}{r^{\Delta_{1}+\Delta_{2}}}
\end{equation}
固定$ r_0$ ,我们得到两点函数形如\footnote{[这里也可以根据Euler齐次函数定理得到微分方程$\frac{\mathrm{d}G^{(2)}}{\mathrm{d}r}r=-\left(\Delta_1+\Delta_2\right)G^{(2)}$从而得到下式。]}
\begin{equation}
	\left\langle\phi_{1}\left(x_{1}\right) \phi_{2}\left(x_{2}\right)\right\rangle=\frac{C_{12}}{\left|x_{1}-x_{2}\right|^{\Delta_{1}+\Delta_{2}}}
\end{equation}
$C_{12} $是常数。特殊共形变换 (1.59) 下,Jacobian是 $|\partial x'/\partial x|=\left(1-2 b x+b^{2} x^{2}\right)^{-d}$ 。在两点函数的共形Ward恒等式中,代入从标度不变性得到的形式 (2.74) ,得到
\begin{equation}
	\begin{aligned} \frac{C_{12}}{\left|x_{1}-x_{2}\right|^{\Delta_{1}+\Delta_{2}}}=& \frac{1}{\left(1-2 b x_{1}+b^{2} x_{1}^{2}\right)^{\Delta_{1}}\left(1-2 b x_{2}+b^{2} x_{2}^{2}\right)^{\Delta_{2}}} \\ & \times \frac{C_{12}}{\left|x_{1}^{\prime}-x_{2}^{\prime}\right|^{\Delta_{1}+\Delta_{2}}} \end{aligned}
\end{equation}
再由两点间距离 $|x_i-x_j| $在特殊共形变换 (1.59) 下的变换行为:
\begin{equation}
	\left|x_{i}^{\prime}-x_{j}^{\prime}\right|=\frac{\left|x_{i}-x_{j}\right|}{\left(1-2 b x_{i}+b^{2} x_{i}^{2}\right)^{\frac{1}{2}}\left(1-2 b x_{j}+b^{2} x_{j}^{2}\right)^{\frac{1}{2}}}
\end{equation}
我们有
\begin{equation}
	\begin{aligned} \frac{C_{12}}{\left|x_{1}-x_{2}\right|^{\Delta_{1}+\Delta_{2}}}=& \frac{1}{\left(1-2 b x_{1}+b^{2} x_{1}^{2}\right)^{\Delta_{1}}\left(1-2 b x_{2}+b^{2} x_{2}^{2}\right)^{\Delta_{2}}} \times \\ & \frac{C_{12}\left(1-2 b x_{1}+b^{2} x_{1}^{2}\right)^{\frac{\Delta_{1}+\Delta_{2}}{2}}\left(1-2 b x_{2}+b^{2} x_{2}^{2}\right) \frac{\Delta_{1}+\Delta_{2}}{2}}{\left|x_{1}-x_{2}\right|^{\Delta_{1}+\Delta_{2}}} \end{aligned}
\end{equation}
这要对任意 $x_1,x_2$ 成立,需要有 $\Delta_{1}=(\Delta_{1}+\Delta_{2})/2=\Delta_{2}$ ,也就是 $\Delta_1=\Delta_2 $。于是,准初级场的两点函数是
\begin{equation}
	\left\langle\phi_{1}\left(x_{1}\right) \phi_{2}\left(x_{2}\right)\right\rangle=\left\{\begin{array}{cc} \frac{C_{12}}{\left|x_{1}-x_{2}\right|^{2 \Delta_{1}}} &, \Delta_{1}=\Delta_{2} \\ 0 &, \Delta_{1} \neq \Delta_{2} \end{array}\right.
\end{equation}

将这个式子应用于 d 维中无质量标量场 $\phi(x)$ 的两点函数。这个标量场的共形维数是$ \Delta=(d-2)/2 $,我们有
\begin{equation}
	\left\langle\phi\left(x_{1}\right) \phi\left(x_{2}\right)\right\rangle=\frac{C}{\left|x_{1}-x_{2}\right|^{d-2}}
\end{equation}
这确实与 (2.59) 中 $d$ 维传播子 $D(x)$ 的幂律行为 $r^{-(d-2)} $一致。

\subsection{三点函数}
接下来考虑准初级场的三点函数
\begin{equation}
	\left\langle\phi_{1}\left(x_{1}\right) \phi_{2}\left(x_{2}\right) \phi_{3}\left(x_{3}\right)\right\rangle=G^{(3)}\left(r_{12}, r_{23}, r_{31}\right)
\end{equation}
由Poincaré不变性,这是两点间距离 $r_{12},r_{23},r_{31} $的函数。一般的共形变换下有
\begin{equation}
	\begin{aligned} \left\langle\phi_{1}\left(x_{1}\right) \phi_{2}\left(x_{2}\right) \phi_{3}\left(x_{3}\right)\right\rangle=& \Omega^{-\Delta_{1}}\left(x_{1}\right) \Omega^{-\Delta_{2}}\left(x_{2}\right) \Omega^{-\Delta_{3}}\left(x_{3}\right) \\ & \times\left\langle\phi_{1}\left(x_{1}^{\prime}\right) \phi_{2}\left(x_{2}^{\prime}\right) \phi\left(x_{3}^{\prime}\right)\right\rangle \end{aligned}
\end{equation}
标度变换 $x'=\alpha x $下, $G^{(3)}$ 满足
\begin{equation}
	G^{(3)}\left(r_{12}, r_{23}, r_{31}\right)=\alpha^{\Delta_{1}+\Delta_{2}+\Delta_{3}} G^{(3)}\left(\alpha r_{12}, \alpha r_{23}, \alpha r_{31}\right)
\end{equation}
与两点函数一样,令 $\alpha r_{12}=r^0_{12}$ ,得到
\begin{equation}
	G^{(3)}\left(r_{12}, r_{23}, r_{31}\right)=\frac{\left(r_{12}^{0}\right)^{\Delta_{1}+\Delta_{2}+\Delta_{3}}}{\left(r_{12}\right)^{\Delta_{1}+\Delta_{2}+\Delta_{3}}} G^{(3)}\left(r_{12}^{0}, \frac{r_{12}^{0} r_{23}}{r_{12}}, \frac{r_{12}^{0} r_{31}}{r_{12}}\right)
\end{equation}
右边的$ r_{23}/r_{12},r_{31}/r_{12} $如果足够小,可以进行展开。于是三点函数形如
\begin{equation}
	G^{(3)}\left(r_{12}, r_{23}, r_{31}\right)=\sum_{a, b, c} \frac{C_{a b c}}{r_{12}^{a} r_{23}^{b} r_{31}^{c}}
\end{equation}
这里,幂次 $a,b,c $满足 $a+b+c=\Delta_{1}+\Delta_{2}+\Delta_{3} $。考虑特殊共形变换下的变换行为,各项要满足
\begin{equation}
\frac{C_{a b c}}{r_{12}^{a} r_{23}^{b} r_{31}^{c}}=\frac{1}{\gamma_{1}^{\Delta_{1}} \gamma_{2}^{\Delta_{2}} \gamma_{3}^{\Delta_{3}}} \frac{C_{a b c} \gamma_{1}^{\frac{a+c}{2}} \gamma_{2}^{\frac{a+b}{2}} \gamma_{3}^{\frac{b+c}{2}}}{r_{12}^{a} r_{23}^{b} r_{31}^{c}}
\end{equation}\
这里, $\gamma_{i}=1-2 b x_{i}+b^{2} x_{i}^{2} $。由此,幂次 $a,b,c $满足
\begin{align} &\Delta_{1}=\frac{a+c}{2}\\ &\Delta_{2}=\frac{a+b}{2}\\ &\Delta_{3}=\frac{b+c}{2} \end{align}
解得
\begin{align} &a=\Delta_{1}+\Delta_{2}-\Delta_{3}\\ &b=\Delta_{2}+\Delta_{3}-\Delta_{1}\\ &c=\Delta_{3}+\Delta_{1}-\Delta_{2}\end{align}
于是,准初级场的三点函数形如
\begin{equation}
	\left\langle\phi_{1}\left(x_{1}\right) \phi_{2}\left(x_{2}\right) \phi\left(x_{3}\right)\right\rangle=\frac{C_{123}}{r_{12}^{\Delta_{1}+\Delta_{2}-\Delta_{3}} r_{23}^{\Delta_{2}+\Delta_{3}-\Delta_{1}} r_{31}^{\Delta_{3}+\Delta_{1}-\Delta_{2}}}
\end{equation}
$C_{123}$ 是常数。

\subsection{四点函数}
准初级场的两点和三点函数,已经从共形不变性确定了。那么四点以上的关联函数呢?考虑四点函数
\begin{equation}
		\left\langle\phi_{1}\left(x_{1}\right) \phi_{2}\left(x_{2}\right) \phi_{3}\left(x_{3}\right) \phi_{4}\left(x_{4}\right)\right\rangle=G^{(4)}\left(r_{12}, r_{13}, r_{14}, r_{23}, r_{24}, r_{34}\right)
\end{equation}\
根据标度变换和特殊共形变换下的变换行为,右边可展开成形如
\begin{equation}
	\sum_{a, b, c, d, e, f} \frac{C_{a b c d e f}}{r_{12}^{a} r_{13}^{b} r_{14}^{c} r_{23}^{d} r_{24}^{e} r_{34}^{f}}
\end{equation}
这里,幂次 $a,b,c,d,e,f $满足
\begin{equation}
	a+b+c+d+e+f=\Delta_{1}+\Delta_{2}+\Delta_{3}+\Delta_{4}
\end{equation}
和
\begin{align} &\Delta_{1}=\frac{a+b+c}{2}\\ &\Delta_{2}=\frac{a+d+e}{2}\\ &\Delta_{3}=\frac{b+d+f}{2}\\ &\Delta_{4}=\frac{c+d+f}{2} \end{align}
解得
\begin{align} &a=\frac{\Delta}{3}-\Delta_{1}-\Delta_{2}+\alpha+\beta\\ &b=\frac{\Delta}{3}-\Delta_{1}-\Delta_{3}-\alpha\\ &c=\frac{\Delta}{3}-\Delta_{1}-\Delta_{4}-\beta\\ &d=\frac{\Delta}{3}-\Delta_{2}-\Delta_{3}-\beta\\ &e=\frac{\Delta}{3}-\Delta_{2}-\Delta_{4}-\alpha\\ &f=\frac{\Delta}{3}-\Delta_{3}-\Delta_{4}+\alpha+\beta \end{align}
这里, $\Delta=\Delta_1+\Delta_2+\Delta_3+\Delta_4 $, $\alpha,\beta$ 是任意参数。因此, (2.94) 成为
\begin{equation}
	\sum_{\alpha, \beta} C_{\alpha, \beta}\left(\frac{r_{12} r_{34}}{r_{13} r_{24}}\right)^{-\alpha}\left(\frac{r_{12} r_{34}}{r_{23} r_{41}}\right)^{-\beta} \sum_{i<j} r_{i j}^{-\Delta_{i}-\Delta_{j}+\Delta / 3}
\end{equation}
不像三点函数,四点函数的形式没能从共形不变性唯一确定,而是形如
\begin{equation}
	\left\langle\phi_{1}\left(x_{1}\right) \phi_{2}\left(x_{2}\right) \phi_{3}\left(x_{3}\right) \phi_{4}\left(x_{4}\right)\right\rangle=F\left(\frac{r_{12} r_{34}}{r_{13} r_{24}}, \frac{r_{12} r_{34}}{r_{23} r_{41}}\right) \sum_{i<j} r_{i j}^{-\Delta_{i}-\Delta_{j}+\Delta / 3}
\end{equation}
这里的
\[
\frac{r_{12} r_{34}}{r_{13} r_{24}}, \frac{r_{12} r_{34}}{r_{23} r_{41}}\]
称为\textbf{交比(cross ratio)},是标度变换和特殊共形变换下的不变量。

$N $点函数也类似,如何依赖于交比$r_{ij}r_{kl}/r_{jk}r_{il} $无法从全局共形不变性确定。不过,在二维,对局域共形变换的共形Ward恒等式将进一步限制关联函数,并控制多点函数的结构。从下章开始,介绍本书的主题,二维共形场论。二维共形场论中的局域共形对称性非常有用,关联函数的结构可仅从对称性确定,不必考虑场的作用量等微观细节。因此,之后的重点将转到对称性及其表示,更加抽象,但希望大家慢慢跟上。



