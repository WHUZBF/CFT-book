\chapter*{序}
\addcontentsline{toc}{chapter}{序}
本书意在面向研究生讲解二维共形场论的基础。

二维共形场论,由Belavin–Polyakov–Zamolodchikov(BPZ)1984年的论文[1]确立,在理论物理的基础中占有一席之地。不仅是由于在超弦理论和二维临界现象中的应用,而影响着物理,也对数学产生了深远的影响。它涵盖的内容广若天际,无论是英文还是日文的,都已有很多广受好评的综述和教科书。

本书不涉及共形场论的最新进展,而是仅限于真正的基础知识。因此,只有同极小模型相关的内容,尤其是尝试给出BPZ论文的详细解读。BPZ的论文是共形场论的“经典”,简洁深刻,常读常新。

本书的第1-3章意在引入共形场论,抓住共形场论的概要。从第4章开始,讲解Virasoro代数的表示,关联函数的计算,环面上的模不变性,边界共形场论和极小模型。

阅读本书需要场论,Lie群和Lie代数等的常识。也会用到复变函数,椭圆函数和特殊函数等。写作这本书时,我参考了场论,Lie群和Lie代数的文献[2][3][4],复变函数论的文献[5][6][7],特别是还有书[8][9][10][11][12][13],综述论文[14][15][16],和论文集[17][18]。

本书未涉及的更深入的主题有,仿射Lie代数及其自由场表示[13],GKO构造[9],W代数和量子Hamiltonian约化[19],$c=1$ 
共形场论[14],有理共形场论[20][21],共形场论的形变[11],超共形代数及其表示论[10]。有关这些主题,可以参考其它的书和综述。共形场论的重要应用之一是超弦理论,本书不提及,可以参考例如[22]。

写作这本书时,松尾泰,鸭下智和横山大辅给的评论很有参考意义。本书能完成也承蒙平势耕介等科学出版社的编辑部成员和伊崎修通的关照。在此表示感谢。

\rightline{伊藤克司}
\rightline{2011年4月}

\section*{参考文献列表}
{\raggedright
[1]A. A. Belavin, A. M. Polyakov and A. B. Zamolodchikov, Nucl. Phys. B 241 (1984) 333.

[2]九後汰一郎,『ゲージ場の量子論I』,培風館,1989.

[3]M. E. Peskin and D. V. Schroeder, An Introduction to Quantum Field Theory, Perseus Books Publ., 1995.

[4]坂井典佑,『場の量子論』,裳華房,2002.

[5]R.V. チャーチル,J.W. ブラウン,『複素関数入門 原書第4 版』,中野実訳,数学書房,1989.

[6]犬井鉄郎,『特殊函数』,岩波全書,1962.

[7]A. フルヴィッツ,R. クーラント,『楕円関数論』,足立恒雄,小松啓一訳,シュプリンガー・フェ アラーク東京,1991.

[8]C. Itzykson and J. M. Drouffe, Statistical Field Theory : Volume 2, Strong Coupling, Monte Carlo Methods, Conformal Field Theory and Random Systems, Cambridge Univ. Press, 1991.

[9]P. Di Francesco, P. Mathieu and D. Sénéchal, Conformal Field Theory, Springer–Verlag, 1997.

[10]R. Blumenhagen and E. Plauschinn, “Introduction to conformal field theory”, Lect. Notes Phys. 779 (2009) 1.

[11]G. Mussardo, Statistical Field Theory : An Introduction to Exactly Solved Models in Statistical Physics, Oxford Univ. Press, 2009.

[12]川上則雄,梁成吉,『共形場理論と1 次元量子系』,岩波書店,1997.

[13]山田泰彦,『共形場理論入門』,培風館, 2006.

[14]P. H. Ginsparg, “Applied Conformal Field Theory”, arXiv:hep-th/9108028.

[15]A. B. Zamolodchikov, Al. B. Zamolodchikov, “Conformal Field Theory and Critical Phenomena in Two-dimensional Systems”, Physics Reviews Vol. 10.4, Harwood, 1989.

[16]J. Cardy, arXiv:0807.3472 [cond-mat.stat-mech].

[17]C. Itzykson, H. Saleur and J. B. Zuber, Conformal Invariance and Applications to Statistical Mechanics, World Scientific, 1988.

[18]P. Goddard and D. I. Olive, Adv. Ser. Math. Phys. 3 (1988) 1.

[19]P. Bouwknegt and K. Schoutens, Phys. Rept. 223 (1993) 183 [arXiv:hep-th/9210010].

[20]A. Tsuchiya and Y. Kanie, Adv. Stud. Pure Math. 16 (1988) 297 [Erratum–ibid. 19 (1989) 675].

[21]G. W. Moore and N. Seiberg, Commun. Math. Phys. 123 (1989) 177.

[22]J. Polchinski, String Theory, Vol. 1, 2, Cambrdige Univ. Press, 1998. (和訳:『ストリン グ理論』,伊藤克司,小竹悟,松尾泰訳,シュプリンガー・フェアラーク東京,2005).}

\section*{搬运说明}

原文为日文,由知乎用户\href{https://www.zhihu.com/people/wo-bei-56}{@笠道梓}翻译,原文刊载于知乎专栏\href{https://www.zhihu.com/column/c_1245739225699885056}{Only for Fun},本人仅仅是将其整理为\LaTeX 文档并修改些许错误,最终著作权由伊藤克司教授所有,翻译版权由\href{https://www.zhihu.com/people/wo-bei-56}{@笠道梓}所有。有些脚注是本人添加的,用[]标出,还有些是原书没有\href{https://www.zhihu.com/people/wo-bei-56}{@笠道梓}标出的,精力有限,不作区分。
